\subsection{Dataset}
\begin{frame}{CHAOS Dataset}
    \citeall{kavur_chaos_2019}
    \begin{itemize}
        \item $512 \times 512 $ CT images of 20 different patients with healthy liver (no tumor, lesions or any other diseases)
        \item For each patient, there is a series of DICOM images ($\sim$100 slices per patient)
        \item A dataset with challenges : partial volume effects, atypical liver shapes, etc.
    \end{itemize}
    % Partial volume effect is an artifact that occurs in CT scans when tissues of widely different absorption are encompassed on the same CT voxel, producing a beam attenuation proportional to the average value of these tissues

    \begin{figure}[H]
        \begin{subfigure}{.3\textwidth}
            \centering
            \includegraphics[width=\textwidth]{issues_ct1.png}
            \caption{Atypical shape}
        \end{subfigure}
        \begin{subfigure}{.3\textwidth}
            \centering
            \includegraphics[width=\textwidth]{issue_ct2.png}
            \caption{Unclear boundary}
        \end{subfigure}
        \begin{subfigure}{.3\textwidth}
            \centering
            \includegraphics[width=\textwidth]{issue_ct3.png}
            \caption{Disconnected parts}
        \end{subfigure}
    \end{figure}
\end{frame}

\subsection{Train FCN}
\begin{frame}{Train FCN}
    Transfer learning with pre-trained FCN-8 weights from \citeauthor*{long_fully_2015}
    \begin{columns}
        \begin{column}{0.5\textwidth}
            \centering
            \includegraphics[width=\textwidth]{FCN8_pretrainedfreeze_lr0005_bs20_weights1x6-0.2.png} \\ 
            (1) Train only the new layers \\
            \tiny{\grey{Batch size of $20$, use of Cosine Annealing Learning Rate Scheduler from lr $= 0.005$}}
        \end{column}

        \begin{column}{.5\textwidth}
            \centering
            \includegraphics[width=\textwidth]{FCN8_pretrainedunfreeze.png} \\ 
            (2) Unfreeze some pre-trained layers {\tiny(the last 9)} \\
            \tiny{\grey{Batch size of $16$, lr $= 0.0001$}}
        \end{column}
    \end{columns}
    \vfill
    \grey{$2005$ training images, $200$ epochs, SGD optimizer with momentum of $.9$} \\ 
    The early stopper was never reached
\end{frame}

\subsection{Solve ACM}
\begin{frame}{Level Set Resolution}

 $\frac{\partial \varphi}{\partial t} = G(\varphi) $
    \begin{algorithm}[H]
        \caption{Level Set Method}
        \begin{algorithmic}
            \Input 
            \Desc{$I$}{(Normalized) Image}
            \Desc{$\varphi_0$}{Signed distance to the initial contour}
            \Desc{$N$}{Number of iterations} \Desc{$\delta t$}{Step size}
            \Desc{$n$}{Re-distancing period}
            \EndInput
            \State \textbf{Initialize} $\varphi^{(0)} \gets \varphi_0$
            \For{$t=0$ to $N$}
                \State Compute $G(\varphi^{(t)})$
                \State \magenta{$\varphi^{(t+1)} \gets \varphi^{(t)} - \delta t\ G(\varphi^{(t)})$} \Comment{Gradient Descent}
                \State Re-distancing $\varphi^{(t+1)}$ every $n$ iterations \Comment{Levelset Re-distancing} % :  computing the signed distance function to the zero level set
            \EndFor
        \end{algorithmic}
    \end{algorithm}
\end{frame}

\subsection{Comparative Analysis: Evaluating Active Contours Models}

\begin{frame}{Comparative Evaluation}
    \framesubtitle{Mean Curvature Motion with \magenta{global} regional Chan Vese forces}

    $$ G(\varphi) = \omega_0\norm{\nabla\varphi}\textrm{div}\left(\frac{\nabla \varphi}{\norm{\nabla\varphi}}\right) + \omega_1 ((I - c_{\textrm{ext}})^2 - (I - c_{\textrm{int}})^2 ) $$

    \begin{columns}
        \begin{column}{0.95\textwidth}    
        $$ c_{\textrm{int}} = \frac{\int_{\Omega}I(x)(1 - H(\varphi(x)))dx}{\int_\Omega 1 - H(\varphi(x)) dx} $$
        $$ c_{\textrm{ext}} = \frac{\int_{\Omega}I(x)H(\varphi(x))dx}{\int_\Omega H(\varphi(x)) dx} $$
        \end{column}
        \begin{column}{0.25\textwidth}
        \hspace*{-2cm}\includegraphics[width=\textwidth]{heaviside_func.png}
        \end{column}
    \end{columns}
    \vspace{0.7cm}
    \centering{\grey{\tiny 1000 iterations, $\omega_0=0.001$, $\omega_1=5$, $\delta t = 0.4$}}

    \includegraphics[width=\textwidth]{solve_acm__CVglobal_w[0.001, 0, 5]_FCNorder2.png}
\end{frame}

\begin{frame}{Comparative Evaluation}
    \framesubtitle{Mean Curvature Motion with \magenta{global} regional Chan Vese forces \magenta{with edge information}}
    \citeall{caselles_geodesic_1997}

    $$ G(\varphi) = \omega_0\magenta{g(I)}\norm{\nabla\varphi}\textrm{div}\left(\frac{\nabla \varphi}{\norm{\nabla\varphi}}\right) + \omega_0\dotp{\magenta{\nabla g}}{\nabla\varphi} +  \omega_1 F_{CV}  $$

    $$ g(I) = \frac{1}{\epsilon + \norm{\nabla\ k_\sigma * I}} $$


    \vspace{0.7cm}
    \centering{\tiny \grey{1000 iterations, $\omega_0=0.001$, $\omega_1=5$, $\delta t = 0.4$}}

    \includegraphics[width=\textwidth]{solve_acm__CVglobal_w[0.001, 0, 5]_FCNorder2_edgeinfo.png}
\end{frame}

\begin{frame}{Comparative Evaluation}
    \framesubtitle{Adding Edge Information}
    \citeall{caselles_geodesic_1997} 

    $$ W = g(I) = \frac{1}{\epsilon + \norm{\nabla\ k_\sigma * I}} $$
    \centering
    \includegraphics[height=.65\textheight]{edge_info.png}
\end{frame}

\begin{frame}{Comparative Evaluation}
    \framesubtitle{Mean Curvature Motion with \magenta{local} regional Chan Vese forces}
    \citeall{lankton_localizing_2008} 

    $$ G(\varphi) = \omega_0\norm{\nabla\varphi}\textrm{div}\left(\frac{\nabla \varphi}{\norm{\nabla\varphi}}\right) + \omega_1 ((I - c_{\textrm{ext}, x})^2 - (I - c_{\textrm{int}, x})^2 ) $$
    $$ B_r(x, y) = \ones_{\norm{x - y} \leq r}$$
    \begin{columns}
        \begin{column}{0.5\textwidth}
            $ c_{\textrm{int}, x} = \frac{\int_{\Omega}B_r(x, y)I(y)(1 - H(\varphi(y)))dy}{\int_\Omega B_r(x, y) (1 - H(\varphi(y))) dy} $
        \end{column}
        \begin{column}{.5\textwidth}
            $
            c_{\textrm{ext}, x} = \frac{\int_{\Omega}B_r(x, y)I(y)H(\varphi(y))dy}{\int_\Omega B_r(x, y) . H(\varphi(y)) dy}
            $
        \end{column}
    \end{columns}
    
    \vspace{0.4cm}
    \centering{\grey{\tiny 200 iterations, $\omega_0=0.1$, $\omega_1=5$, $\delta t = 0.4$, $r=10$}}

    \includegraphics[width=\textwidth]{solve_acm__CVlocal_r10_w[0.1, 0, 5]_FCNorder2.png}
\end{frame}

\begin{frame}{Comparative Evaluation}
    \framesubtitle{Mean Curvature Motion with \magenta{local} regional Chan Vese forces \magenta{with edge information}}
    \citeall{caselles_geodesic_1997}

    $$ G(\varphi) = \omega_0\magenta{g(I)}\norm{\nabla\varphi}\textrm{div}\left(\frac{\nabla \varphi}{\norm{\nabla\varphi}}\right) + \omega_0\dotp{\magenta{\nabla g}}{\nabla\varphi} +  \omega_1 F_{CV,x}  $$

    $$ g(I) = \frac{1}{\epsilon + \norm{\nabla\ k_\sigma * I}} $$


    \vspace{0.7cm}
    \centering{\tiny \grey{1000 iterations, $\omega_0=0.1$, $\omega_1=5$, $\delta t = 0.4$, $r=10$}}

    \includegraphics[width=\textwidth]{solve_acm__CVlocal_r10_w[0.1, 0, 5]_FCNorder2_edgeInfo.png}
\end{frame}

\begin{frame}{Comparative Evaluation}
    \framesubtitle{Mean Curvature Motion with \magenta{global} regional Chan Vese forces \magenta{with additional $F_{\textrm{FCN}}$}}
    \citeall{guo_automatic_2019} 

    $$ G(\varphi) = \omega_0 \norm{\nabla \varphi}\kappa + \omega_1F_{CV} + \omega_2 F_{\textrm{FCN}}$$
    \begin{columns}
        \begin{column}{.5\textwidth}
    Polynomial formulation of order $p$
    $ F_{\textrm{FCN}} = \textrm{sign}(L(x, y)) \abs{L(x, y)}^p \vec{n}$
        \end{column}
        \begin{column}{.5\textwidth}
            Exponential formulation
    $F_{\textrm{FCN}} = \alpha\ .\ \textrm{sign}(L(x, y)) \textrm{e}^{\abs{L(x, y)}}\vec{n}$
        \end{column}
    \end{columns}
    \vspace{0.7cm}
    \centering{\tiny \grey{1000 iterations, $\omega_0=0.01$, $\omega_1=5$, $\omega_2=1$, $\delta t = 0.4$, $p=2$}}
    \includegraphics[width=\textwidth]{solve_acm__CVglobal_w[0.01, 1, 5]_FCNorder2.png}
    
\end{frame}

\begin{frame}{Comparative Evaluation}
    \framesubtitle{External constraint $F_{\textrm{FCN}}$ forces}

    \begin{center}
    \tiny \grey{Polynomial function with $p=2$} 

    \includegraphics[height=.8\textheight]{f_fcn.png}
    \end{center}

\end{frame}

\begin{frame}{Comparative Evaluation}
    \framesubtitle{Mean Curvature Motion with \magenta{local} regional Chan Vese forces \magenta{with additional $F_{\textrm{FCN}}$}}

    $$ G(\varphi) = \omega_0 \norm{\nabla \varphi}\kappa + \omega_1F_{CV, x} + \omega_2 F_{\textrm{FCN}}$$

    \vspace{0.7cm}
    \centering{\tiny \grey{1000 iterations, $\omega_0=0.01$, $\omega_1=5$, $\omega_2=1$, $\delta t = 0.4$, $r=10$, $p=2$}}
    \includegraphics[width=\textwidth]{solve_acm__CVlocal_w[0.01, 1, 5]_FCNorder2.png}
\end{frame}

\subsection{A Pathological Case}
\begin{frame}{A Pathological Case}
    \centering
\includegraphics[height=.45\textheight]{ex_bad_case.png}
\includegraphics[height=.5\textheight]{ex_bad_case_levelset.png}
\end{frame}

\begin{frame}{A Pathological Case - Result}
    Mean Curvature Motion with global Chan Vese forces and polynomial FCN forces \\ 
    \centering{\tiny \grey{1000 iterations, $\omega_0=0.01$, $\omega_1=5$, $\omega_2=1$, $\delta t = 0.4$, $p=2$}} 

    \centering
    \includegraphics[height=.4\textheight]{bad_case_f_fcn.png}
    \includegraphics[width=\textwidth]{bad_case__CVglobal__w[0.01, 0.5, 5]_FCNorder2.png}
\end{frame}
